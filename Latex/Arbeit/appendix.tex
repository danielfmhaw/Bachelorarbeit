% -*- coding: utf-8 -*-

% Ausgabe des Literaturverzeichnisses; ohne weitere Optionen werden nur die
% Bücher und Artikel ausgegeben, die in der Arbeit auch zitiert werden.
\printbibliography

% markiert den Anfang des Anhangs
\appendix

% ein Kapitel, das nicht numeriert, aber trotzdem ins Inhaltsverzeichnis
% aufgenommen wird
\addchap{Anhang}
% TODO (Daniel): update here se-app
Hier beginnt der Anhang.
Siehe die Anmerkungen zur Sinnhaftigkeit eines Anhangs in Abschnitt%~\ref{sec-app} auf Seite~\pageref{sec-app}.

Der Anhang kann wie das eigentliche Dokument in Kapitel und Abschnitte unterteilt werden.
Der Befehl \verb|\appendix| sorgt im Wesentlichen nur für eine andere Nummerierung.

% neue Seite
\clearpage

% keine Seitenzahl
\thispagestyle{empty}

% keine Nummerierung, keine Aufnahme ins Inhaltsverzeichnis
\section*{Eigenständigkeitserklärung}

% Hier müssen Sie natürlich den Titel der Arbeit sowie Ort und Datum ersetzen:
Hiermit versichere ich, dass ich die vorliegende Bachelorarbeit mit dem Titel
\begin{center}
  \textbf{Performance - Optimierung von Datenbanken}
\end{center}
selbstständig und nur mit den angegebenen Hilfsmitteln verfasst habe.
Alle Passagen, die ich wörtlich aus der Literatur oder aus anderen Quellen wie z.\,B. Internetseiten übernommen habe, habe ich deutlich als Zitat mit Angabe der Quelle kenntlich gemacht.

\vspace{2cm}
% TODO (Daniel): update this here
Hamburg, 13.\ März 2025
