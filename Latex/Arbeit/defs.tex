% Jede LaTeX-Datei beginnt mit der Dokumentenklasse.  Für diese Vorlage wurde
% die Klasse "scrreprt" von KOMA-Script gewählt, die in etwa der
% Standardklasse "report" entspricht, allerdings wesentlich mehr Möglichkeiten
% bietet und im gewissen "moderner" ist.  Eine sehr ausführliche Dokumentation
% zu KOMA-Script findet man unter der folgenden Adresse:
% http://mirrors.ctan.org/macros/latex/contrib/koma-script/doc/scrguide.pdf
\documentclass[
  % die Schriftgröße - sollten Sie nicht ändern
  fontsize=12pt,
  % das Papierformat, also DIN A4
  paper=A4,
  % Literaturverzeichnis ins Inhaltsverzeichnis
  bibliography=totoc,
  % andere Verzeichnisse ebenfalls ins Inhaltsverzeichnis
  listof=totoc,
  % abgesetzte Formeln linksbündig
  fleqn,
  % für die Satzspiegelkonstruktion - siehe KOMA-Doku
  DIV=12,
  % Bindekorrektur (linker Rand) - evtl. anpassen
  BCOR=1mm,
  % die im Text verwendeten Sprachen (u.a. für das Paket babel); die
  % letztgenannte (!) Sprache ist die Standardsprache; "n"german steht für die
  % neue Rechtschreibung
  english,ngerman,
  % weil (s.u.) das Paket geometr verwendet wird
  usegeometry,
  % wie Absätze gesetzt werden: ohne Einzug, halbe Zeile Abstand
  parskip=half-
]{scrreprt}

% Beschriftungen für Tabellen kommen linksbündig über die Tabelle
\KOMAoption{captions}{tableheading,nooneline}
\setcaptionalignment[figure]{c}
\setcaptionalignment[table]{l}

% wird für die Titelseite benötigt
\usepackage{geometry}

% Standardpaket für Lokalisation, siehe Option "ngerman" oben
\usepackage{babel}
% Laden von optimierten Trennmustern
\babelprovide[hyphenrules=ngerman-x-latest]{ngerman}

% Standardpaket für mathematische Zusatzfunktionen; wenn Sie keine
% mathematischen Formeln brauchen, können Sie diese Zeile löschen
\usepackage{amsmath}

% die Hauptschrift Libertinus
\usepackage{libertinus-otf}
% die "Schreibmaschinenschrift" Anonymous Pro, angepasst
\usepackage{AnonymousPro}
\setmonofont{AnonymousPro}[Scale=MatchLowercase,FakeStretch=0.85]

% etwas größerer Zeilenabstand als im Buchsatz
\linespread{1.1}

% Paket für Feinkorrekturen an der Typographie, das für ein ausgewogeneres
% Schriftbild sorgt
\usepackage{microtype}

% Paket für kontextsensitive Anführungszeichen
\usepackage{csquotes}
% Shortcut, damit aus dem eigentlich falschen Zeichen " richtige
% Anführungszeichen je nach Sprache werden
\MakeOuterQuote{"}

% Paket, das den Befehl \includegraphics ermöglicht
\usepackage{graphicx}

% komfortablere Aufzählungen als in Standard-LaTeX; ein Beispiel findet man in
% chap3.tex
\usepackage{enumitem}

% Paket für mehr als die üblichen Standardfarben
\usepackage[dvipsnames]{xcolor}
% Definition der "Hausfarben" der HAW
\definecolor{haw}{HTML}{003CA0}
\definecolor{haw2}{HTML}{0096D2}
\definecolor{haw3}{HTML}{A0BEDC}
\definecolor{darkgreen}{rgb}{0, 0.5, 0}
\definecolor{background}{HTML}{EEEEEE}
\definecolor{delim}{RGB}{20,105,176}
\colorlet{punct}{red!60!black}
\colorlet{numb}{magenta!60!black}

% typographisch anspruchsvolle Tabellen; siehe chap3.tex
\usepackage{booktabs}

% zum Erstellen des Literaturverzeichnisses; der gängige Stil APA ist hier
% bereits eingestellt
% TODO (Daniel): change bib stile
\usepackage[style=ieee]{biblatex}
% eine Beispieldatei für ein Literaturverzeichnis
\addbibresource{demo.bib}

% für die Erzeugung der Grafiken in chap3.tex; wenn Sie PGF/TikZ nicht
% verwenden wollen, können Sie diese Zeilen entfernen
\usepackage{tikz}
% Zusatzbibliotheken für TikZ, die in den genannten Beispielen verwendet
% werden
\usetikzlibrary{calc,intersections,angles,3d}

% für die Erzeugung des Codeblocks in chap3.tex; wenn in Ihrer Arbeit keine
% Codeblöcke vorkommen, können Sie diese Zeilen entfernen
\usepackage{listings}

% see https://tex.stackexchange.com/questions/83085/how-to-improve-listings-display-of-json-files
\lstdefinelanguage{json}{
    basicstyle=\normalfont\ttfamily,
    numbers=left,
    numberstyle=\scriptsize,
    stepnumber=1,
    numbersep=8pt,
    showstringspaces=false,
    breaklines=true,
    frame=lines,
    backgroundcolor=\color{background},
    literate=
    *{0}{{{\color{numb}0}}}{1}
        {1}{{{\color{numb}1}}}{1}
        {2}{{{\color{numb}2}}}{1}
        {3}{{{\color{numb}3}}}{1}
        {4}{{{\color{numb}4}}}{1}
        {5}{{{\color{numb}5}}}{1}
        {6}{{{\color{numb}6}}}{1}
        {7}{{{\color{numb}7}}}{1}
        {8}{{{\color{numb}8}}}{1}
        {9}{{{\color{numb}9}}}{1}
        {:}{{{\color{punct}{:}}}}{1}
        {,}{{{\color{punct}{,}}}}{1}
        {\{}{{{\color{delim}{\{}}}}{1}
        {\}}{{{\color{delim}{\}}}}}{1}
        {[}{{{\color{delim}{[}}}}{1}
        {]}{{{\color{delim}{]}}}}{1},
}
% see https://tex.stackexchange.com/questions/152829/how-can-i-highlight-yaml-code-in-a-pretty-way-with-listings
\newcommand\YAMLcolonstyle{\color{red}\mdseries}
\newcommand\YAMLkeystyle{\color{black}\bfseries}
\newcommand\YAMLvaluestyle{\color{blue}\mdseries}

\makeatletter

\newcommand\language@yaml{yaml}

\expandafter\expandafter\expandafter\lstdefinelanguage
\expandafter{\language@yaml}
{
    keywords={true,false,null,y,n},
    keywordstyle=\color{darkgray}\bfseries,
    basicstyle=\YAMLkeystyle,
    sensitive=false,
    comment=[l]{\#},
    morecomment=[s]{/*}{*/},
    commentstyle=\color{purple}\ttfamily,
    stringstyle=\YAMLvaluestyle\ttfamily,
    moredelim=[l][\color{orange}]{\&},
    moredelim=[l][\color{magenta}]{*},
    moredelim=**[il][\YAMLcolonstyle{:}\YAMLvaluestyle]{:},
    morestring=[b]',
    morestring=[b]",
    literate =    {---}{{\ProcessThreeDashes}}3
        {>}{{\textcolor{red}\textgreater}}1
        {|}{{\textcolor{red}\textbar}}1
        {\ -\ }{{\mdseries\ -\ }}3,
}

\lst@AddToHook{EveryLine}{\ifx\lst@language\language@yaml\YAMLkeystyle\fi}
\makeatother
\newcommand\ProcessThreeDashes{\llap{\color{cyan}\mdseries-{-}-}}

% Anpassung des Erscheinungsbildes des Codeblocks; mehr dazu in der
% Dokumentation des Pakets "listings"
\lstdefinestyle{mystyle}{
    backgroundcolor=\color{gray!20},
    keywordstyle=\color{haw2},
    numberstyle=\footnotesize\color{haw},
    basicstyle=\ttfamily\small,
    captionpos=t,
    frame=single,
    framerule=0pt,
    keepspaces=true,
    numbers=left,
    numbersep=6pt,
    belowcaptionskip=1em,
    aboveskip=\bigskipamount,
}

\lstdefinestyle{custom_daniel}{
    backgroundcolor=\color{gray!20},
    basicstyle=\ttfamily\small,
    keywordstyle=\color{haw2},
    commentstyle=\color{gray},
    numberstyle=\footnotesize\color{haw},
    stringstyle=\color{darkgreen},
    numbers=left,
    breaklines=true,
    keepspaces=true,
    showspaces=false,
    showstringspaces=false,
}
\lstset{style=mystyle}
% damit es "Codeblock" und nicht "Listing" heißt
\renewcommand{\lstlistingname}{Codeblock}

% für die Verlinkung innerhalb des PDF-Dokuments, für PDF-Lesezeichen und
% PDF-Metadaten; dieses Paket sollte üblicherweise immer als letztes geladen
% werden
\usepackage[colorlinks=true,allcolors=haw,hyperfootnotes=false,pageanchor=true,linktoc=all]{hyperref}

% für die Druckversion können Sie die obige Zeile durch die folgende ersetzen,
% damit Links nicht blau dargestellt werden:
% \usepackage[draft]{hyperref}

% Metadaten des PDF-Dokumentes; setzen Sie hier Ihren eigenen Namen sowie den
% Titel Ihrer Arbeit ein
\hypersetup{pdfauthor={Prof. Dr. Edmund Weitz}}
\hypersetup{pdftitle={Handreichung zur Formatierung von Bachelorarbeiten}}

\usepackage{subcaption}