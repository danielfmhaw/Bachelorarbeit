% In dieser Umgebung wird die Titelseite separat vom restlichen Text gesetzt
\begin{titlepage}
  % andere Seitenränder als im Rest der Arbeit
  \newgeometry{lmargin=2cm,tmargin=7mm,rmargin=5mm,bmargin=1cm}
  % die "Hausfarbe" der HAW; diese und die folgenden Einstellungen sind lokal
  % und gelten nur innerhalb der Umgebung "titlepage"
  \color{haw}
  % Blocksatz für die Titelseite deaktivieren
  \raggedright
  % Logo rechtsbündig setzen
  \hfill\includegraphics[width=7cm]{PNGs/General/HAW_Marke_RGB_300dpi}\\

  % vertikaler Abstand
  \vspace{5cm}

  % Wahl der "Hausschrift" Open Sans der HAW, die als Schrift auf Ihrem
  % Rechner installiert sein muss
  \setmainfont{Open Sans}
  % etwas kleiner als üblich
  \small
  % fett und in Majuskeln
  \textbf{BACHELORARBEIT}

  % vertikaler Abstand
  \vspace{8mm}

  % der Titel der Arbeit als "Seite in der Seite"; natürlich müssen Sie hier
  % Ihren Titel eintragen
  \begin{minipage}{0.8\linewidth}
    % Wahle der zweiten "Hausschrift" der HAW, die ebenfalls auf Ihrem Rechner
    % bereits vorhanden sein muss
    \setmainfont{Martel Heavy}
    % ziemlich große Schrift
    \LARGE
    % [1mm] steht jeweils für einen etwas größeren Durchschuss
    Performance -\\[1mm]
    Optimierung\\[1mm]
    von Datenbanken\\
    % am Ende noch ein waagerechter Strich, das CD will es so...
    \,\rule{11mm}{1.2mm}
  \end{minipage}

  % vertikaler Abstand, überraschenderweise
  \vspace{1cm}

  % hier korrektes Datum und Ihren Namen eingeben
  % TODO (Daniel): update this
  vorgelegt am 13.\ März 2025\\
  Daniel Freire Mendes

  % letzter vertikaler Abstand für heute
  \vspace{5cm}

  % noch eine "Seite in der Seite", etwas nach rechts geschoben
  \hspace*{37mm}
  \begin{minipage}{0.5\linewidth}
    % Namen und Titel der beiden Prüfer eintragen
    \begin{tabular}{@{}ll}
      Erstprüferin: & Prof. Dr. Stefan Sarstedt\\[-.3mm]
      Zweitprüfer: & Prof. Dr. Olaf Zukunft \\
    \end{tabular}\\

    % noch ein horizontaler Strich
    \,\rule{9mm}{1mm}\\[1.5mm]

    \textbf{HOCHSCHULE FÜR ANGEWANDTE}\\
    \textbf{WISSENSCHAFTEN HAMBURG}\\
    Department Informatik\\
    Berliner Tor 7\\
    20099 Hamburg
  \end{minipage}
\end{titlepage}
% setzt die Geometrie wieder auf die Standardwerte zurück
\restoregeometry

% für die Seite mit dem Abstract keine Seitenzahl ausgeben
\thispagestyle{empty}
\section*{Zusammenfassung}

% Hier ersetzen Sie bitte die vorhandenen Texte durch Ihre eigenen
% Zusammenfassungen
% Aufbau: Motivation → Problemstellung → Lösungsansatz → Ergebnisse → Schlussfolgerungen

Relationale Datenbanken sind ein essenzieller Bestandteil moderner IT-Systeme und bilden die Grundlage für zahlreiche Anwendungen, die täglich von Millionen von Nutzern verwendet werden.
Mit wachsender Datenmenge steigen jedoch die Antwortzeiten von Abfragen, was die Systemnutzung erschwert.
Die Herausforderung besteht darin, geeignete Optimierungsstrategien zu finden, die sowohl Lese- als auch Schreiboperationen effizient gestalten und eine hohe Skalierbarkeit gewährleisten.
Diese Arbeit untersucht verschiedene Ansätze zur Optimierung der Performance, darunter Datentypen, Indexierung, Views, Partitionierung und Replikation.
Zur Analyse der Auswirkungen dieser Methoden wird das Tool Sysbench für Leistungsmessungen eingesetzt.
Die Ergebnisse zeigen, dass die Wahl des kleinstmöglichen Datentyps und die Verwendung von Not Null-Spalten die Effizienz verbessern, indem sie Speicherplatz sparen.
Hash-Indizes sind besonders bei exakten Schlüsselvergleichen effektiv, wohingegen B-Baum-Indizes vielseitigere Einsatzmöglichkeiten bieten.
Materialisierte Sichten bieten Performancevorteile durch gespeicherte Abfrageergebnisse, im Gegensatz dazu liefern virtuelle Sichten Echtzeitdaten, müssen jedoch bei jedem Zugriff die Abfrage neu ausführen und sind daher langsamer.
Bei großen Datenmengen kann Partitionierung eine effektive Lösung darstellen, während Replikation die Lastverteilung insbesondere bei hoher CPU-Last verbessert.
Es gibt keine universelle Lösung, aber je nach Anwendungsfall können geeignete Konzepte ausgewählt, optimiert und auch miteinander kombiniert werden.

% Zum Wechseln der Sprache siehe den Kommentar in chap3.tex
{
  \begin{otherlanguage}{english}
    \section*{Abstract}
    Relational databases are an essential component of modern IT systems and form the foundation for numerous applications used daily by millions of users.
    However, as data volumes grow, query response times increase, making system usage more challenging.
    The challenge lies in identifying suitable optimization strategies that make both read and write operations efficient while ensuring high scalability.
    This paper examines various approaches to performance optimization, including data types, indexing, views, partitioning and replication.
    The impact of these methods is analyzed through benchmarking with the Sysbench tool.
    The results show that choosing the smallest possible data type and using Not Null columns optimizes performance by saving storage space.
    Hash indexes are particularly effective for exact key comparisons, while B-tree indexes offer more versatile applications.
    Materialized views provide performance benefits by storing query results, whereas virtual views deliver real-time data but execute the query anew with each access, making them slower.
    For large datasets, partitioning can be an effective solution, with replication improving load distribution, especially under high CPU load.
    There is no universal solution, but depending on the use case, suitable concepts can be selected, optimized and even combined.
  \end{otherlanguage}
}
