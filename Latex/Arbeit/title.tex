% In dieser Umgebung wird die Titelseite separat vom restlichen Text gesetzt
\begin{titlepage}
  % andere Seitenränder als im Rest der Arbeit
  \newgeometry{lmargin=2cm,tmargin=7mm,rmargin=5mm,bmargin=1cm}
  % die "Hausfarbe" der HAW; diese und die folgenden Einstellungen sind lokal
  % und gelten nur innerhalb der Umgebung "titlepage"
  \color{haw}
  % Blocksatz für die Titelseite deaktivieren
  \raggedright
  % Logo rechtsbündig setzen
  \hfill\includegraphics[width=7cm]{PNGs/General/HAW_Marke_RGB_300dpi}\\

  % vertikaler Abstand
  \vspace{5cm}

  % Wahl der "Hausschrift" Open Sans der HAW, die als Schrift auf Ihrem
  % Rechner installiert sein muss
  \setmainfont{Open Sans}
  % etwas kleiner als üblich
  \small
  % fett und in Majuskeln
  \textbf{BACHELORARBEIT}

  % vertikaler Abstand
  \vspace{8mm}

  % der Titel der Arbeit als "Seite in der Seite"; natürlich müssen Sie hier
  % Ihren Titel eintragen
  \begin{minipage}{0.8\linewidth}
    % Wahle der zweiten "Hausschrift" der HAW, die ebenfalls auf Ihrem Rechner
    % bereits vorhanden sein muss
    \setmainfont{Martel Heavy}
    % ziemlich große Schrift
    \LARGE
    % [1mm] steht jeweils für einen etwas größeren Durchschuss
    Performance -\\[1mm]
    Optimierung\\[1mm]
    von Datenbanken\\
    % am Ende noch ein waagerechter Strich, das CD will es so...
    \,\rule{11mm}{1.2mm}
  \end{minipage}

  % vertikaler Abstand, überraschenderweise
  \vspace{1cm}

  % hier korrektes Datum und Ihren Namen eingeben
  % TODO (Daniel): update this
  vorgelegt am 26. März 2022\\
  Daniel Freire Mendes

  % letzter vertikaler Abstand für heute
  \vspace{5cm}

  % noch eine "Seite in der Seite", etwas nach rechts geschoben
  \hspace*{37mm}
  \begin{minipage}{0.5\linewidth}
    % Namen und Titel der beiden Prüfer eintragen
    \begin{tabular}{@{}ll}
      Erstprüferin: & Prof. Dr. Stefan Sarstedt\\[-.3mm]
      Zweitprüfer: & Prof. Dr. Olaf Zukunft \\
    \end{tabular}\\

    % noch ein horizontaler Strich
    \,\rule{9mm}{1mm}\\[1.5mm]

    \textbf{HOCHSCHULE FÜR ANGEWANDTE}\\
    \textbf{WISSENSCHAFTEN HAMBURG}\\
    Department Informatik\\
    Berliner Tor 7\\
    20099 Hamburg
  \end{minipage}
\end{titlepage}
% setzt die Geometrie wieder auf die Standardwerte zurück
\restoregeometry

% für die Seite mit dem Abstract keine Seitenzahl ausgeben
\thispagestyle{empty}
% TODO (Daniel): write text here
\section*{Zusammenfassung}

% Hier ersetzen Sie bitte die vorhandenen Texte durch Ihre eigenen
% Zusammenfassungen
% Aufbau: Motivation → Problemstellung → Lösungsansatz → Ergebnisse → Schlussfolgerungen
TO-DO weiterverbessern: bisher nur eine erste Skizze der Zusammenfassung.

In modernen Datenbanksystemen, insbesondere bei den weit verbreiteten relationalen Datenbanken, ist die Performance ein entscheidender Faktor.
Je komplexer die Abfragen und je größer das Datenvolumen, desto länger kann die Ausführungsdauer bestimmter Abfragen sein.
Eine der größten Herausforderungen besteht darin, eine geeignete Strategie für das Problem auszuwählen.

Abhängig von zusätzlichen Zielen neben der Performance, wie der Verteilung der Daten, gibt es unterschiedliche Lösungskonzepte.
Es geht auch darum einschätzen zu können, inwiefern sich die Veränderungen bezüglich der Lese- als auch Schreibgeschwindigkeit auswirken.
Dafür benutzen wir ein Benchmark-Tool sowie verschiedene Tools zur Visualisierung der Ergebnisse.

In der Bachelorarbeit werden verschiedene Lösungsansätze, wie Datentypen, Indexierung, Views, Partitionen und Replikation näher erläutert.
Zu jedem Thema wird ein Beispiel gegeben, das als Orientierung dient und die Auswirkungen der jeweiligen Änderungen verdeutlicht.
Unser Fokus liegt auf dem Datenbankmanagementsystem MySQL, jedoch wird im Kapitel zu Views die native Implementierung von materialisierten Views in PostgreSQL analysiert.

Die vorgestellten Optimierungsansätze dieser Arbeit bieten wertvolle Erkenntnisse für Datenbankadministratoren, die leistungsfähige Lösungen in verschiedenen Anwendungsbereichen implementieren möchten.
\newline
\newline
\newline
Der Arbeit beginnt mit einer kurzen Beschreibung ihrer zentralen Inhalte, in der die Thematik und die wesentlichen Resultate skizziert werden.
Diese Beschreibung muss sowohl in deutscher als auch in englischer Sprache vorliegen und sollte eine Länge von etwa 150 bis 250 Wörtern haben.
Beide Versionen zusammen sollten nicht mehr als eine Seite umfassen.
Die Zusammenfassung dient u.\,a.\ der inhaltlichen Verortung im Bibliothekskatalog.

% Zum Wechseln der Sprache siehe den Kommentar in chap3.tex
{
  \begin{otherlanguage}{english}
    % TODO (Daniel): write text here
    \section*{Abstract}

    The thesis begins with a brief summary of its main contents, outlining the subject matter and the essential findings.
    This summary must be provided in German and in English and should range from 150 to 250 words in length.
    Both versions combined should not comprise more than one page.
    Among other things, the abstract is used for library classification.
  \end{otherlanguage}
}
