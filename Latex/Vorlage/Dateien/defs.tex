% Jede LaTeX-Datei beginnt mit der Dokumentenklasse.  Für diese Vorlage wurde
% die Klasse "scrreprt" von KOMA-Script gewählt, die in etwa der
% Standardklasse "report" entspricht, allerdings wesentlich mehr Möglichkeiten
% bietet und im gewissen "moderner" ist.  Eine sehr ausführliche Dokumentation
% zu KOMA-Script findet man unter der folgenden Adresse:
% http://mirrors.ctan.org/macros/latex/contrib/koma-script/doc/scrguide.pdf
\documentclass[
  % die Schriftgröße - sollten Sie nicht ändern
  fontsize=12pt,
  % das Papierformat, also DIN A4
  paper=A4,
  % Literaturverzeichnis ins Inhaltsverzeichnis
  bibliography=totoc,
  % andere Verzeichnisse ebenfalls ins Inhaltsverzeichnis
  listof=totoc,
  % abgesetzte Formeln linksbündig
  fleqn,
  % für die Satzspiegelkonstruktion - siehe KOMA-Doku
  DIV=12,
  % Bindekorrektur (linker Rand) - evtl. anpassen
  BCOR=1mm,
  % die im Text verwendeten Sprachen (u.a. für das Paket babel); die
  % letztgenannte (!) Sprache ist die Standardsprache; "n"german steht für die
  % neue Rechtschreibung
  english,ngerman,
  % weil (s.u.) das Paket geometr verwendet wird
  usegeometry,
  % wie Absätze gesetzt werden: ohne Einzug, halbe Zeile Abstand
  parskip=half-
]{scrreprt}

% Beschriftungen für Tabellen kommen linksbündig über die Tabelle
\KOMAoption{captions}{tableheading,nooneline}
\setcaptionalignment[figure]{c}
\setcaptionalignment[table]{l}

% wird für die Titelseite benötigt
\usepackage{geometry}

% Standardpaket für Lokalisation, siehe Option "ngerman" oben
\usepackage{babel}
% Laden von optimierten Trennmustern
\babelprovide[hyphenrules=ngerman-x-latest]{ngerman}

% Standardpaket für mathematische Zusatzfunktionen; wenn Sie keine
% mathematischen Formeln brauchen, können Sie diese Zeile löschen
\usepackage{amsmath}

% die Hauptschrift Libertinus
\usepackage{libertinus-otf}
% die "Schreibmaschinenschrift" Anonymous Pro, angepasst
\usepackage{AnonymousPro}
\setmonofont{AnonymousPro}[Scale=MatchLowercase,FakeStretch=0.85]

% etwas größerer Zeilenabstand als im Buchsatz
\linespread{1.1}

% Paket für Feinkorrekturen an der Typographie, das für ein ausgewogeneres
% Schriftbild sorgt
\usepackage{microtype}

% Paket für kontextsensitive Anführungszeichen
\usepackage{csquotes}
% Shortcut, damit aus dem eigentlich falschen Zeichen " richtige
% Anführungszeichen je nach Sprache werden
\MakeOuterQuote{"}

% Paket, das den Befehl \includegraphics ermöglicht
\usepackage{graphicx}

% komfortablere Aufzählungen als in Standard-LaTeX; ein Beispiel findet man in
% chap3.tex
\usepackage{enumitem}

% Paket für mehr als die üblichen Standardfarben
\usepackage[dvipsnames]{xcolor}
% Definition der "Hausfarben" der HAW
\definecolor{haw}{HTML}{003CA0}
\definecolor{haw2}{HTML}{0096D2}
\definecolor{haw3}{HTML}{A0BEDC}

% typographisch anspruchsvolle Tabellen; siehe chap3.tex
\usepackage{booktabs}

% zum Erstellen des Literaturverzeichnisses; der gängige Stil APA ist hier
% bereits eingestellt
\usepackage[style=apa]{biblatex}
% eine Beispieldatei für ein Literaturverzeichnis
\addbibresource{demo.bib}

% für die Erzeugung der Grafiken in chap3.tex; wenn Sie PGF/TikZ nicht
% verwenden wollen, können Sie diese Zeilen entfernen
\usepackage{tikz}
% Zusatzbibliotheken für TikZ, die in den genannten Beispielen verwendet
% werden
\usetikzlibrary{calc,intersections,angles,3d}

% für die Erzeugung des Codeblocks in chap3.tex; wenn in Ihrer Arbeit keine
% Codeblöcke vorkommen, können Sie diese Zeilen entfernen
\usepackage{listings}
% Anpassung des Erscheinungsbildes des Codeblocks; mehr dazu in der
% Dokumentation des Pakets "listings"
\lstdefinestyle{mystyle}{
    backgroundcolor=\color{gray!20},
    keywordstyle=\color{haw2},
    numberstyle=\footnotesize\color{haw},
    basicstyle=\ttfamily\small,
    captionpos=t,
    frame=single,
    framerule=0pt,
    keepspaces=true,
    numbers=left,
    numbersep=6pt,
    belowcaptionskip=1em,
    aboveskip=\bigskipamount,
}
\lstset{style=mystyle}
% damit es "Codeblock" und nicht "Listing" heißt
\renewcommand{\lstlistingname}{Codeblock}

% für die Verlinkung innerhalb des PDF-Dokuments, für PDF-Lesezeichen und
% PDF-Metadaten; dieses Paket sollte üblicherweise immer als letztes geladen
% werden
\usepackage[colorlinks=true,allcolors=haw,hyperfootnotes=false,pageanchor=true,linktoc=all]{hyperref}

% für die Druckversion können Sie die obige Zeile durch die folgende ersetzen,
% damit Links nicht blau dargestellt werden:
% \usepackage[draft]{hyperref}

% Metadaten des PDF-Dokumentes; setzen Sie hier Ihren eigenen Namen sowie den
% Titel Ihrer Arbeit ein
\hypersetup{pdfauthor={Prof. Dr. Edmund Weitz}}
\hypersetup{pdftitle={Handreichung zur Formatierung von Bachelorarbeiten}}
